\documentclass[a4j,12pt,landscape]{jarticle}
\topmargin=3mm \headheight=0mm \headsep=0mm
\textheight=170mm 
\textwidth=220mm
\evensidemargin=0mm \oddsidemargin=0mm
\footskip=0mm
\parindent=0cm

\usepackage[all]{xy}
\usepackage{amssymb}
\usepackage{amscd}
\usepackage{amsmath}
\usepackage{graphics}
\usepackage{graphicx}
\usepackage{amsthm}

\pagestyle{empty}

\def\modulo{{\rm mod}}
\def\F2{{\mathbb F}_2}
\def\wt{{\rm wt}}
\def\wo{{\rm wt}_o}
\def\wf{{\rm wt}_f}
\def\UL{{\rm ul}}
\def\bx{{{\mathbf x}}}
\def\by{{{\mathbf y}}}
\def\bw{{{\mathbf w}}}
\def\bu{{{\mathbf u}}}


\title{}
\author{}
\date{\today}

\begin{document}
{\Huge

\vspace*{2cm}
\begin{center}
{\bf Hearty Twister: a new random number generator}
\vspace{1cm}

Hiroshi Haramoto (Hiroshima Univ.), \\
Francois Panneton(Caisse centrale Desjardins), \\
Takuji Nishimura(Yamagata Univ.), \\
and \\
Makoto Matsumoto (Hiroshima Univ.).
\end{center}
\newpage

\newpage
\noindent
{\bf 1. Pseudo Random Number Genarator}\\

In this talk, PRNG is a (finite state) automaton consisting of:
\begin{itemize}
\item State set $S$. 
\item Transition function $f:S \to S$.
\item Initial state $s_0 \in S$.
\item (No imput considered.)
\end{itemize}
Change the state as follows:
$$
s_0 \mapsto s_1:= f(s_0) \mapsto s_2:=f(s_1) \mapsto \cdots.
$$

\newpage
Let $O$ be the set of output symbols.

(In this talk, we assume $O$=32-bit integers.)

\vskip 1cm
At each state $s$, 
the output is determined by 
\begin{itemize}
\item an output function $o:S \to O$.
\end{itemize}
The output sequence becomes
$$
o(s_0),o(s_1),o(s_2),\ldots.
$$
We use this as a pseudo-random sequence.
\newpage
\begin{center}
\includegraphics[width=0.7\linewidth]{figure01.eps}
\\
Figure 1: Automaton.
\end{center}

 
\newpage
\noindent
{\bf 2. Feedbacked Shift Regiter}\\
In the above automaton, 
its period is bounded  \\
by the number of the states $\#(S)$.

\vskip 5mm
We want large $S$ (like 20000 bits), \\
which admits fast computation of $f$ \\
in a software implementation.

\vskip 5mm
Feedbacked Shift Register (FSR) permits $f$ \\
with $O(1)$-time complexity.

\newpage
Feedbacked Shift Register.
\begin{itemize}
\item
$S=$an array of words(=32-bit integers) of length $N$:
$$
S = \{(\bw_0,\ldots,\bw_{N-1})| \bw_i \mbox{: 32-bit integer}\}.
$$
\item
$f:(\bw_0,\ldots,\bw_{N-1}) \mapsto (\bw_1,\ldots,\bw_{N-1}, \by)$,

where $\by$ is given by a function $g$
$$\by:=g(\bw_0,\ldots,\bw_{N-1}).$$
\end{itemize}

Round robin reduces the computation of $f$ to that of $g$. 

For high speed generation, $g$ looks at only a few words \\
in the state array.
\newpage

\begin{center}
\includegraphics[width=0.7\linewidth]{figure02.eps}
\\
Figure 2: Feedbacked Shift Register.
\\
\end{center}
\newpage
\noindent
{\bf 3 Linear Generator}

We identify: 
\begin{itemize}
\item 
a bit $\{0,1\}$ with the two element field $\F2$, and
\item 
a word (32-bit integer) with horizontal vector $\F2^{32}$.
\item 
Regard $S$ as an $\F2$-vector space.
\end{itemize}
An automaton is said to be {\em $\F2$-linear} if 
the transition function $f$ and 
the output function $o$ are both $\F2$-linear. \\
~~~We consider only this kind of automata, with 
justification:
\begin{itemize}
\item the period is computable
\item some kinds of assurance on the distribution \\
(such as the dimension of equidistribution) can be given
\item can be converted to a non-linear sequence 
by using a non-linear transform. 
\end{itemize}

\newpage
\noindent
{\bf 4. Examples of Linear FSR}

{\bf GFSR} ('73, Lewis-Payne)
$$
g(\bw_0,\ldots,\bw_{N-1}) = g(\bw_0, \bw_M) = \bw_0 + \bw_M.
$$
The function $g$ reads only two words, so this is called \\
a two-tap FSR.

Period is $2^N-1$.

\newpage
{\bf Twisted GFSR} ('92, '94, Matsumoto-Kurita)
$$
g(\bw_0,\ldots,\bw_{N-1}) = g(\bw_0, \bw_M) = \bw_0A + \bw_M,
$$
where $A$ is a ($32\times 32$)-matrix over $\F2$, defined by
$$
\bx A = 
\left\{\begin{array}{ll}
  \mbox{shiftright}({\bf x}) & (\mbox{if LSB of $\bx$ is 0}) \\
  \mbox{shiftright}({\bf x})\oplus{\bf a} & 
  (\mbox{if LSB of $\bx$ is 1}),
\end{array}\right.
$$
where ${\bf a}$ is a constant vector.
Period is $2^{32N}-1$. 
\begin{center}
\includegraphics[width=0.6\linewidth]{figure03.eps}
\\
Figure 3: Twisted GFSR
\end{center}
\newpage
{\bf Mersenne Twister} ('98 Matsumoto-Nishimura)
$$
g(\bw_0,\ldots,\bw_{N-1}) = g(\bw_0, \bw_1, \bw_M) = (\bw_0|\bw_1)A + \bw_M,
$$
where $(\bw_0|\bw_1)$ denotes
the concatenation of $32-r$ MSBs of $\bw_0$ and $r$ LSBs of $\bw_1$,
and $A$ is the same as above.

Period is $2^{32N-r}-1$, chosen to be a Mersenne prime.

\newpage
\noindent
{\bf 5. Zero excess state problem.}

FSRs with small number of taps has the following problem: \\
If the state space contains too many zeroes, \\
then the tendency continues for long time.
\begin{center}
\includegraphics[width=0.6\linewidth]{figure03.eps}
\\
\vskip -5mm
Figure 3': Twisted GFSR as a FSR with few taps.
\end{center}
 From Figure 3', one can see that if the almost 
all words are zero, then the next incorporated word
tends to be zero.



\newpage
A possible improvement: 
As in Figure 4, add one more tap.
Then, once $\bw_{N-1}$ becomes nonzero, the state
quickly recovers from zero-excess state.

\begin{center}
\includegraphics[width=0.7\linewidth]{figure04.eps}
\\
Figure 4: A possible improvement for zero-excess
\end{center}

\newpage 
\noindent
{\bf 6 Hearty Twister}

In the previous Figure 4, 
$\bw_{N-1}$ is both read and written, \\
so it is faster to prepare one variable $\bu$ instead of
$\bw_{N-1}$. 

~~~Then, Figure 4 is the same with Figure 5.
\begin{center}
\includegraphics[width=0.7\linewidth]{figure05.eps}
\\
Figure 5: separate the variable $\bw_{N-1}$
\\
\end{center}

\newpage
Now we may feedback to $\bw_{N-2}$ not only from $\bu$;
thus obtain Figure~6, the {\bf Hearty Twister}.

\begin{center}
\includegraphics[width=0.7\linewidth]{figure06.eps}
\\
Figure 6: Hearty Twister.
\\
\end{center}

\newpage
Figure 7 discribes the blood circulation system of \\
Fish (FSR type) and Frog (Hearty Twister type). \\
The variable $\bu$ serves as a lung, which generates new 1's quickly. 

\begin{center}
\includegraphics[width=0.7\linewidth]{figure07.eps}\\
Figure 7: Blood circulation system: \\
FSR type (left) and Hearty Twister (right).
\\
\end{center}


\newpage
Figure 8 shows an implementation HT800 with period $2^{800}-1$.
The function $T_{<< n}$ is defined by 
$\bx \mapsto \bx + {\mathrm{shiftleft}(\bx, n)}$.

($>> n$ shift right)

\begin{center}
\includegraphics[width=0.7\linewidth]{figure08.eps}
\\
Figure 8: Detailed description of HT800.
\end{center}

\newpage
Below is WELL generator (Panneton-L'Ecuyer-Matsumoto, '05). \\
It has an excellent performance at the cost of speed.

\begin{center}
\includegraphics[width=0.7\linewidth]{figure09.eps}
\\
Figure 9: WELL prn generator.
\\
Each $T$ denotes (distinct) linear transformation.
\end{center}

\newpage
\noindent
{\bf 7 Comparison of performances}
\\
{\bf 7-1} Compared Generators 
\begin{description}
\item MT19937: Mersenne Twister with period $2^{19937}-1$ ('98).
\item TT800: Tempered Twisted GFSR with period $2^{800}-1$. \\
Matsumoto-Kurita ('92, '94)
\item XS800: A FSR-type generator XORSHIFT by Marsaglia ('03).
 An optimal parameter searched by Panneton-L'Ecuyer('04).
\item WELL1024: WELL generator by Panneton-L'Ecuyer-Matsumoto with period $2^{1024}-1$ ('05).
\item WELL44497b: WELL generator with period $2^{44497}-1$ ('05).
\item HT800: Hearty Twister with period $2^{800}-1$.
\item HT1279: Hearty Twister with period $2^{1279}-1$.
\end{description}
\newpage

\begin{center}
\includegraphics[width=0.7\linewidth]{figure10.eps}
\\
Figure 10: XOR-SHIFT XS800
\\
Proposed by Marsaglia (2003), a best parameter
found by Panneton and L'Ecuyer (2004). 
\end{center}

\newpage
\noindent
{\bf 7-2} High dimensional distribution.

{\bf Definition.} 
A periodic sequence with period $P$
$$\bx_0, \bx_1, \ldots, \bx_{P-1}, \bx_P=\bx_0, \ldots$$
of $v$-bit integers is said to be {\em $k$-dimensionally equidistributed}
if any $kv$-bit pattern occurs equally often as a $k$-tuple
$$
(\bx_i, \bx_{i+1}, \ldots, \bx_{i+k-1})
$$
for a period $i=0,\ldots, P-1$. 

(The all-zero pattern occurs once less often.)

\newpage
A periodic sequence of 32-bit integers is said to be
{\em $k$-dimensionally equidistributed with $v$-bit accuracy}
if the most significant $v$ bits of each integer are
$k$-dimensionally equidistributed. 

We denote by $k(v)$ the maximum such $k$. 

\vskip 5mm
We have an upperbound 
$$
k(v) \leq \lfloor \log_2 (P+1) / v \rfloor, 
$$
and define the dimension defect as the summation of the gaps 
$$
\Delta_1 := \sum_{v=1}^{32}(\lfloor \log_2 (P+1) / v \rfloor -k(v)).
$$

%%%*** change 05/05/10
\newpage
\begin{center}
%\label{table:dist}
\begin{tabular}{|c||c|c|c|c|c|}
\hline
{\Huge Generator}& {\Huge $\Delta_1$} 
  & {\Huge Linux} & {\Huge MacOSX} & {\Huge FreeBSD} & {\Huge Windows} \\ \hline \hline
{\Huge MT19937} & {\Huge 6750}
  & {\Huge 0.92} & {\Huge 1.63} & {\Huge 1.41} & {\Huge 1.42} \\ \hline
{\Huge TT800}  & {\Huge 261} 
  & {\Huge 0.82} & {\Huge 1.51} & {\Huge 1.28} & {\Huge 1.14} \\ \hline
{\Huge XS800} & {\Huge 186}
  & {\Huge 0.79} & {\Huge 1.44} & {\Huge 0.96} & {\Huge 0.89} \\ \hline
{\Huge WELL1024} & {\Huge 0}
  & {\Huge 1.48} & {\Huge 2.23} & {\Huge 1.34} & {\Huge 1.44} \\ \hline
{\Huge WELL44497b} & {\Huge 0}
  & {\Huge 2.13} & {\Huge 3.48} & {\Huge 2.38} & {\Huge 2.52} \\ \hline
{\Huge HT800} & {\Huge 72} 
  & {\Huge 0.95} & {\Huge 1.51} & {\Huge 1.10} & {\Huge 1.02} \\\hline
{\Huge HT1279} & {\Huge 113}
  & {\Huge 0.80} & {\Huge 1.36} & {\Huge 1.06} & {\Huge 0.92} \\ \hline
\end{tabular}
\\
\vskip 5mm
Table 1. Comparison of the dimension defects $\Delta_1$ 
\\
and the time (sec.) to generate $10^8$ prns (gcc -O2). 
\end{center}
%\end{table}

WELL has the optimal equidistribution property, 
but slower than the other generators. 

\newpage
\noindent
{\bf 7-3 Recovery time from zero-excess state}\\

\begin{enumerate}
\item Choose an initial state $\in S$ with only one bit being 1.
\item Generate $k$ pseudorandom numbers, and discard them.
\item Compute the ratio of 1's among the 
next 3200 bits 
\\
(i.e. in the next 100 pseudorandom numbers).
\item Let $\gamma_k$ be the average of the ratio over
all such initial states.
\end{enumerate}

\newpage

\begin{center}
\includegraphics[width=0.7\linewidth]{gamma4.eps}
\\
Figure~11. 
$\gamma_k \quad (k=0,\ldots,1000)$
\\
The order of the recovery speed: \\
WELL1024 (Salmon Pink, Best), HT800 (Blue), \\
HT1279 (Purple), XS800 (Red), TT800 (Black).
\end{center}


\newpage
\noindent
{\bf 7-4} Coin-tossing gambling test
\begin{itemize}
\item Simulate 32 times coin-tossings by a 32-bit integer.
\item Observe the number of 1s in the past ($\dim S$)-times coin-tossings. 
\item Decide to bet or not, on ``all zero for the next 16 tossings.''

 If win, get $2^{16}$ dollars. If lost, lose $1$ dollar. 
\item Choose ``the best strategy'' using the MacWilliams identity
(Haramoto-Matsumoto-Nishimura, submitted 2004).
\item Compute the expected value of the gained money.
\end{itemize}

%%%*** change 05/05/10
\newpage

The result is as follows:
\begin{center}
\begin{tabular}{|c|c|}
\hline
{\Huge Genarator} &{\Huge Expected \$ per bet}  \\ \hline \hline
{\Huge TT800} & {\Huge $1.155\times 10^{-5}$} \\ \hline
{\Huge XS800} & {\Huge $0.511\times 10^{-5}$} \\ \hline
{\Huge WELL1024} & {\Huge $0.3372 \times 10^{-151}$} \\ \hline 
{\Huge HT800} & {\Huge $0.335\times 10^{-11}$}  \\ \hline
{\Huge HT1279} & {\Huge $0.363 \times 10^{-14}$} \\ \hline
\hline
\end{tabular}
\\
\vskip 5mm
Table 2. The expected gained money per one bet. \\
MT19937 and WELL44497b exceeded \\
the capacity of our program. 

\end{center}


\newpage
\noindent
{\bf 8 HT with Mersenne Prime Period}\\
\begin{itemize}
\item To realize HT1279, we searched for $f$ whose
characteristic polynomial is reducible: 
$(t+1)\times \phi(t)$ with $\phi(t)$ irreducible of degree 1279.
\item We found HT4423, whose characteristic polynomial is \\
$(t+1)\times \mbox{(degree 4423)}$. The $k(v)$'s are not yet computed. 
\item We have been searching for higher degree Hearty Twisters,
e.g. HT19937 with $\mbox{(degree 31)} \times \mbox{(degree 19937)}$
but not yet succeeded (because of some bugs).
\end{itemize}

\newpage
\noindent
{\bf 9 Concluding Remarks}
\begin{itemize}
\item We proposed Hearty Twister. 
\item Faster than Twister GFSR TT800 and WELL.
\item Faster recovery from zero-excess states \\
than TT800 (but inferior to WELL). 
\item Better $k(v)$ than TT800 \\
(but worse than WELL).
\item Better coin-tossing distribution than TT800 \\
(but worse than WELL).
\end{itemize}
We conclude that Hearty Twister is a good compromise
between speed and quality. 

%\newpage
%
%remark that these improvements (except for the speed) \\
%might be somehow illusive:
%
%\begin{itemize}
%\item 
%The dimension defects hardly cause problems \\
%in MonteCarlo simulation. 
%
%\item 
%The zero-excess states hardly occur \\
%in the actual random number generation process \\
%(except for those caused by a bad initialization scheme).
%\end{itemize}

}
\end{document}




